%====================================================================================================
% Computação Móvel no Auxílio de Alunos com Necessidades Especiais
%====================================================================================================
% TCC
%----------------------------------------------------------------------------------------------------
% Autor			: Diego Dalto e Marcio Ariani Feitosa
% Orientador		: Kleber Kruger
% Instituição 		: UFMS - Universidade Federal do Mato Grosso do Sul
% Departamento		: CPCX - Sistema de Informação
%----------------------------------------------------------------------------------------------------
% Data de criação	: 12 de Novembro de 2017
%====================================================================================================
\chapter{Dificuldades Enfrentadas pelos alunos com NNEs e a Importância das Tecnologias de Informação e Comunicação} \label{cap:funTeorica}

No presente capítulo é apresentado as principais dificuldades na educação enfrentado pelos alunos com NEEs. Também será apresentado a importância da implantação das TICs como formas de auxílio no aprendizado e no desenvolvimento dos alunos com algum tipo de necessidade especial. Apesar do presente trabalho abordar o desenvolvimento de uma plataforma educacional, nas próximas seções serão abordadas as diferentes deficiências apresentadas pelos alunos.

\section{Dificuldades Enfrentadas na Educação dos alunos com NNEs} 

Bergonzoni e Belletti \cite{bergonzoni2006dificuldades} mencionam que estudos mostram que o fornecimento de um excelente suporte e de assistência regulada ao desempenho da criança é a possibilidade de tornar o aluno capaz de solucionar problemas e tarefas, tendo como resultado alcançado o desempenho potencial, aquele que vai além do desempenho real.

De acordo com Bergonzoni e Belletti \cite{bergonzoni2006dificuldades}: 
\begin{quote} 
Dificuldade de Aprendizagem (D.A.) é um termo geral que se refere a um grupo heterogêneo de transtornos que se manifestam por dificuldades significativas na aquisição e uso da escuta, fala, leitura, escrita, raciocínio ou habilidades em uma ou várias áreas de estudo. Esses transtornos são intrínsecos ao individuo, supondo-se devido à disfunção do sistema nervoso central, e podem ocorrer ao longo do ciclo vital (p.89).
\end{quote}

\newpage

Podem existir, junto com as dificuldades de aprendizagem, problemas nas condutas de auto-regulação, percepção social e interação social, mas não constituem, por si próprias, uma dificuldade de aprendizagem \cite{bergonzoni2006dificuldades}. 
Diante da dificuldade de aprendizagem, a implantação de plataformas educacionais ricas em estímulos e exercícios favorece, indubitavelmente, o desenvolvimento do aprendizado dos alunos com NEEs. 

Segundo Fernandes e Viana \cite{fernandes2009alunos}, esses alunos apresentam, normalmente, impedimentos de longo prazo, de natureza física, mental, intelectual ou sensorial, que, em interação com diversos obstáculos, podem restringir sua participação efetiva na escola e na sociedade. 

As pessoas com transtornos invasivos do desenvolvimento (autismo)\footnote{O autismo se caracteriza por apresentar alterações na comunicação, interação social e pela presença de padrões de comportamento inalterável. O autista tende a demonstrar: resistência a alterações na rotina, conduta diferente ao ambiente social e tendencia a movimentos repetitivos (Gomes e Mendes, 2010).} e transtorno de déficit de atenção e hiperatividade também necessitam de uma assistência educacional diferenciada.
Importante ressaltar que, alunos com NEEs também tem direito a desenvolver o seu potencial, assegurado por legislação nacional e internacional, podendo colaborar de modo ativo para o progresso artístico e científico de sua nação \cite{fernandes2009alunos}.

A implantação de plataformas educacionais para o auxílio dos alunos com NEEs, tem como um dos objetivos, verificar a qualidade de aprendizagem através de avaliações diagnosticas, sendo esta uma alternativa viável e necessária para estimular não somente o aprendizado, mas também a inteligencia dos alunos.

\section{Implantação das TICs no auxílio da Educação} 


\section{Os Diferentes Tipos de Plataformas Educacionais} 