%====================================================================================================
% Computação Móvel no Auxílio de Alunos com Necessidades Especiais
%====================================================================================================
% TCC
%----------------------------------------------------------------------------------------------------
% Autor			: Diego Dalto e Marcio Ariani Feitosa
% Orientador		: Kleber Kruger
% Instituição 		: UFMS - Universidade Federal do Mato Grosso do Sul
% Departamento		: CPCX - Sistema de Informação
%----------------------------------------------------------------------------------------------------
% Data de criação	: 12 de Novembro de 2017
%====================================================================================================

\chapter{Introdução} \label{Cap:Introducao}

Segundo Fernandes \cite{fernandez2013mobile}, estudantes que precisam de uma educação especial possuem dificuldades para desenvolver habilidades cognitivas e adquirir novos conhecimentos. Eles também precisam melhorar seu comportamento, comunicação e relacionamentos com os diferentes tipos de ambientes. De acordo com Valete \cite{hasselbring2000use}, as crianças com deficiência (física, mental, auditiva ou visual) ou com deficit de atenção TEM COM ACENTO dificuldades que limitam sua capacidade de interagir com o mundo. Estas dificuldades podem impedir que estas crianças desenvolvam habilidades que formam a base do seu processo de aprendizagem.

As dificuldades de aprendizagem e interação acabam sendo agravadas ainda mais quando associadas a uma carência de estímulos, principalmente em situações de limitações econômicas e sociais. Como resultado, tendem a gerar posturas de passividade diante da realidade em que vivem. Diante dessas dificuldades, a introdução das Tecnologias de Informação e Comunicação (TIC) na educação é um processo de reforma que, como diversos outros processos semelhantes, implica tempo, meios e uma maior atuação dos educadores numa multiplicidade de fatores, conforme menciona Giroto \cite{giroto2012tecnologias}. Atualmente, com todas as representações sociais e culturais que temos em nossa sociedade, vemos uma profunda modificação nestes aspectos com o avanço das TICs. Mediante essas tecnologias, o desenvolvimento de plataformas educacionais inteiramente personalizáveis oferecem benefícios que ajuda a moldar o processo de aprendizagem para diferentes aspectos cognitivos, principalmente quando adaptadas as necessidades de cada usuário.

Conforme Fernandes \cite{fernandez2013mobile}, as plataformas educacionais proporcionam ambientes que contém exercícios que podem ser personalizados em níveis de conteúdo e interface dos usuários, mediante a um projeto centrado principalmente nos requisitos dos acadêmicos e que possa ser facilmente entendidos não apenas pelos alunos, mas também pelos professores.

\newpage

\section{Justificativa}

Atualmente as plataformas educacionais existentes oferecem diversos exercícios e vídeos de aprendizagem personalizados, como, por exemplo, a plataforma \textit{Khan Academy}, disponível no endereço \url{https://khanacademy.org}, que utiliza tecnologias adaptativas de ponta que identificam os pontos fortes e lacunas no aprendizado dos alunos.

Segundo Mejias \cite{mejias2006teaching}, as plataformas educacionais são uma ferramenta para aumentar as habilidades sociais e colaborar com o desenvolvimento do aluno, um meio para facilitar a troca de informações e aprendizagem. Algumas plataformas educacionais integram ferramentas de comunicação como bate-papo, e-mail, fórum, quadro de avisos e calendário de atividades. Assim, tanto o professor quanto o aluno podem ter seus próprios cenários de trabalho, com conteúdos e atividades didáticas que podem ser elaboradas por instituições ou pelos próprios professores, que ficarão disponíveis para a comunidade docente para uso, sem limitações, sem problemas de direitos autorais e com grande facilidade para reedição e adaptação às necessidades didáticas de cada aluno.

De acordo com Guerra \cite{Guerra}, estas plataformas, além de oferecerem ao professor um repositório com grande quantidade de atividades sobre variados temas, também geram estatísticas sobre o desempenho dos alunos, possibilitando ao professor, gerenciar suas turmas, por meio de diagnósticos precisos de quais atividades devem ser enfatizadas, conforme o desempenho individual dos alunos.

ENTRETANTO, A MAIORIA DAS PLATAFORMAS EXISTENTES FOCAM NO APRENDIZADO DAS MATERIAS DA GRADE BASICA CURRICULAR DE ENSINO DE NIVEL FUNDAMENTAL E MEDIO, FICANDO ESTAS CRIANCAS DISTANTES DO APRENDIZADO DA COMPUTACAO.

Portanto, o presente trabalho TRATA-SE DE uma continuação do Trabalho de Conclusão de Curso (TCC) realizado pela discente Noemia Francisca Guerra, QUE CONSISTIU no PROJETO DE uma plataforma educacional que, além de oferecer exercícios de aprendizado, disponibiliza MECANISMOS PARA QUE professores E alunos TENHAM acesso aos exercícios e relatórios gerados na plataforma. POREM, POR SE TRATAR DE UMA PLATAFORMA MUITO AMPLA, AQUI VC CITA QUE O SEU SE JUSTIFICA PORQUE TENTA TRAZER OS RETARDADO PARA A COMPUTACAO. Os professores também podem criar e administrar os exercícios na plataforma. Também será disponibilizado interatividade entre alunos e professores, acompanhamento pedagógico através de relatórios e gráficos e lições de casa digitais.

\section{Objetivos Gerais} \label{Sec:Objetivos}

O objetivo deste trabalho é projetar uma plataforma educacional para a web, utilizando um web service que poderá ser utilizado em dispositivos móveis, \textit{desktop} ou qualquer outro meio que tenha acesso a web. Esta plataforma deve permitir a integração de alunos, professores, escolas e responsáveis em um único ambiente. Dentre as várias funções que o ambiente irá oferecer, será possível gerar relatórios com dados estatísticos para que avaliações de desempenho dos alunos possam ser realizadas.


\subsection{Objetivo Específicos} \label{Sec:ObjetivoGeral}		

O presente trabalho possui como objetivos específicos:

\begin{itemize}
	
	\item Item 1.
	conteúdo...
\end{itemize}

• Projetar e desenvolver uma plataforma educacional a fim de auxiliar os professores e alunos
no processo de aprendizagem;

• A plataforma deve oferecer aos professores dados estatísticos acerca dos desempenhos de seus alunos;

• Proporcionar ao aluno atividades lúdicas e desafiadoras para melhorar o processo de aprendizagem;

• Proporcionar ao aluno práticas de ensino que despertem o interesse pela tecnologia;

• Avaliar a capacidade de leitura, escrita, letramento e interpretação de texto por meio da plataforma educacional.


\section{Organização da Proposta} \label{Sec:Organizacao}

As dificuldades da educação e a implantação das TICs como ferramentas de apoio pedagógico são tratados no AUTOREF Capítulo 2, dividido em três seções: na Seção 2.1, são apresentadas as diversas dificuldades enfrentadas pelos Alunos com Necessidades Educacionais Especiais (ANNEs), no processo de educação; na Seção 2.2, a implantação das TICs no apoio da educação e os benefícios tragos por essas tecnologias; e na Seção 2.3, apresenta-se de forma sucinta, as principais plataformas educacionais existentes no mercado.
No Capítulo 3, ...

A metodologia de engenharia de software, as diversas tecnologias e ferramentas utilizadas no desenvolvimento do presente trabalho são apresentadas no Capítulo 4. No Capítulo 5 são apresentados os diagramas de casos de uso, de classes e de sequência realizados neste trabalho.

Por último, no Capítulo 6 são apresentadas as considerações finais deste trabalho.





